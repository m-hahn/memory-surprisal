\documentclass[11pt,letterpaper]{article}

\usepackage{showlabels}
\usepackage{fullpage}
\usepackage{pslatex}
%\usepackage{latexsym}
\usepackage[english]{babel}
\usepackage[utf8]{inputenc}
\usepackage{amsmath}
\usepackage{bm}
\usepackage{graphicx}
\usepackage{tikz}
\usepackage{xcolor}
\usepackage{url}
%\usepackage[colorinlistoftodos]{todonotes}
\usepackage{rotating}
\usepackage{natbib}
\usepackage{amssymb}

\usepackage{tikz-dependency}
\usepackage{longtable}


\newcommand{\R}[0]{\mathbb{R}}
\newcommand{\E}[0]{\mathbb{E}}
\newcommand{\Ff}[0]{\mathcal{F}}

\usepackage{multirow}

\newcommand{\soft}[1]{}
\newcommand{\nopreview}[1]{}
\newcommand\comment[1]{{\color{red}#1}}
\newcommand\mhahn[1]{{\color{red}(#1)}}
\newcommand\note[1]{{\color{red}(#1)}}
\newcommand\jd[1]{{\color{red}(#1)}}
\newcommand\rljf[1]{{\color{red}(#1)}}
\newcommand{\key}[1]{\textbf{#1}}

\usepackage{amsthm}

\newcommand{\thetad}[0]{{\theta_d}}
\newcommand{\thetal}[0]{{\theta_{LM}}}

\newcounter{theorem}
\newtheorem{proposition}[theorem]{Proposition}
\newtheorem{thm}[theorem]{Theorem}
\newtheorem{corollary}[theorem]{Corollary}
\newtheorem{question}[theorem]{Question}
\newtheorem{example}[theorem]{Example}
\newtheorem{lemma}[theorem]{Lemma}


\frenchspacing
%\def\baselinestretch{0.975}

%\emnlpfinalcopy
%\def\emnlppaperid{496}

\title{Supplementary Information for: Crosslinguistic Word Orders Enable an Efficient Tradeoff between Memory and Surprisal}
\author{Michael Hahn, Judith Degen, Richard Futrell}
\date{2018}

\begin{document}

\maketitle






\begin{figure}[!htbp]
\includegraphics[width=0.5\textwidth]{../code/visualize_neural/figures/full-REAL-listener-surprisal-memory-HIST_z_byMem_onlyWordForms_boundedVocab.pdf}
\caption{Histogram}\label{fig:hist-real}
\end{figure}


\begin{table}[!htbp]
\begin{longtable}{l|ll||l|llllllllllllll}
	Language & Training & Held-Out & 	Language & Training & Held-Out\\ \hline
Afrikaans  &  1,315  &  194  &  Indonesian  &  4,477  &  559  \\
Amharic  &  974  &  100  &  Italian  &  17,427  &  1,070  \\
Arabic  &  21,864  &  2,895  &  Japanese  &  7,164  &  511  \\
Armenian  &  514  &  50  &  Kazakh  &  947  &  100  \\
Bambara  &  926  &  100  &  Korean  &  27,410  &  3,016  \\
Basque  &  5,396  &  1,798  &  Kurmanji  &  634  &  100  \\
Breton  &  788  &  100  &  Latvian  &  4,124  &  989  \\
Bulgarian  &  8,907  &  1,115  &  Maltese  &  1,123  &  433  \\
Buryat  &  808  &  100  &  Naija  &  848  &  100  \\
Cantonese  &  550  &  100  &  North Sami  &  2,257  &  865  \\
Catalan  &  13,123  &  1,709  &  Norwegian  &  29,870  &  4,639  \\
Chinese  &  3,997  &  500  &  Persian  &  4,798  &  599  \\
Croatian  &  7,689  &  600  &  Polish  &  6,100  &  1,027  \\
Czech  &  102,993  &  11,311  &  Portuguese  &  17,995  &  1,770  \\
Danish  &  4,383  &  564  &  Romanian  &  8,664  &  752  \\
Dutch  &  18,310  &  1,518  &  Russian  &  52,664  &  7,163  \\
English  &  17,062  &  3,070  &  Serbian  &  2,935  &  465  \\
Erzya  &  1,450  &  100  &  Slovak  &  8,483  &  1,060  \\
Estonian  &  6,959  &  855  &  Slovenian  &  7,532  &  1,817  \\
Faroese  &  1,108  &  100  &  Spanish  &  28,492  &  3,054  \\
Finnish  &  27,198  &  3,239  &  Swedish  &  7,041  &  1,416  \\
French  &  32,347  &  3,232  &  Thai  &  900  &  100  \\
German  &  13,814  &  799  &  Turkish  &  3,685  &  975  \\
Greek  &  1,662  &  403  &  Ukrainian  &  4,506  &  577  \\
Hebrew  &  5,241  &  484  &  Urdu  &  4,043  &  552  \\
Hindi  &  13,304  &  1,659  &  Uyghur  &  1,656  &  900  \\
Hungarian  &  910  &  441  &  Vietnamese  &  1,400  &  800  \\

\end{longtable}
	\caption{Languages, with the number of training and held-out sentences available.}\label{tab:corpora}
\end{table}

\begin{table}[!htbp]
\begin{longtable}{l|ll||l|llllllllllllll}
	Language & Base. & Real & Language & Base. & Real \\ \hline
Afrikaans  &  13  &  10  &  Indonesian  &  11  &  11  \\
Amharic  &  137  &  10  &  Italian  &  10  &  10  \\
Arabic  &  11  &  10  &  Japanese  &  25  &  15  \\
Armenian  &  140  &  76  &  Kazakh  &  11  &  10  \\
Bambara  &  25  &  29  &  Korean  &  11  &  10  \\
Basque  &  15  &  10  &  Kurmanji  &  338  &  61  \\
Breton  &  35  &  14  &  Latvian  &  308  &  178  \\
Bulgarian  &  14  &  10  &  Maltese  &  30  &  24  \\
Buryat  &  26  &  18  &  Naija  &  214  &  10  \\
Cantonese  &  306  &  32  &  North Sami  &  335  &  194  \\
Catalan  &  11  &  10  &  Norwegian  &  12  &  10  \\
Chinese  &  21  &  10  &  Persian  &  25  &  12  \\
Croatian  &  30  &  17  &  Polish  &  309  &  35  \\
Czech  &  18  &  10  &  Portuguese  &  15  &  55  \\
Danish  &  33  &  17  &  Romanian  &  10  &  10  \\
Dutch  &  27  &  10  &  Russian  &  20  &  10  \\
English  &  13  &  11  &  Serbian  &  26  &  11  \\
Erzya  &  846  &  167  &  Slovak  &  303  &  27  \\
Estonian  &  347  &  101  &  Slovenian  &  297  &  80  \\
Faroese  &  27  &  13  &  Spanish  &  14  &  10  \\
Finnish  &  83  &  16  &  Swedish  &  31  &  14  \\
French  &  14  &  11  &  Thai  &  45  &  19  \\
German  &  19  &  13  &  Turkish  &  13  &  10  \\
Greek  &  16  &  10  &  Ukrainian  &  28  &  18  \\
Hebrew  &  11  &  10  &  Urdu  &  17  &  10  \\
Hindi  &  11  &  10  &  Uyghur  &  326  &  175  \\
Hungarian  &  220  &  109  &  Vietnamese  &  303  &  12  \\

\end{longtable}
	\caption{Samples drawn per language according to the precision-dependent stopping criterion.}\label{tab:samples}
\end{table}



\begin{table}[!htbp]
\begin{longtable}{l|lll||l|lllllllllllllll}
	Language & Mean & Lower & Upper & Language & Mean & Lower & Upper \\ \hline
Afrikaans  &  1.0  &  1.0  &  1.0  &  Indonesian  &  1.0  &  1.0  &  1.0  \\
Amharic  &  1.0  &  1.0  &  1.0  &  Italian  &  1.0  &  1.0  &  1.0  \\
Arabic  &  1.0  &  1.0  &  1.0  &  Japanese  &  1.0  &  1.0  &  1.0  \\
Armenian  &  0.92  &  0.87  &  0.97  &  Kazakh  &  1.0  &  1.0  &  1.0  \\
Bambara  &  1.0  &  1.0  &  1.0  &  Korean  &  1.0  &  1.0  &  1.0  \\
Basque  &  1.0  &  1.0  &  1.0  &  Kurmanji  &  0.93  &  0.88  &  0.98  \\
Breton  &  1.0  &  1.0  &  1.0  &  Latvian  &  0.49  &  0.4  &  0.57  \\
Bulgarian  &  1.0  &  1.0  &  1.0  &  Maltese  &  1.0  &  1.0  &  1.0  \\
Buryat  &  1.0  &  1.0  &  1.0  &  Naija  &  1.0  &  0.99  &  1.0  \\
Cantonese  &  0.96  &  0.86  &  1.0  &  North Sami  &  0.37  &  0.3  &  0.44  \\
Catalan  &  1.0  &  1.0  &  1.0  &  Norwegian  &  1.0  &  1.0  &  1.0  \\
Chinese  &  1.0  &  1.0  &  1.0  &  Persian  &  1.0  &  1.0  &  1.0  \\
Croatian  &  1.0  &  1.0  &  1.0  &  Polish  &  0.1  &  0.04  &  0.17  \\
Czech  &  1.0  &  1.0  &  1.0  &  Portuguese  &  1.0  &  1.0  &  1.0  \\
Danish  &  1.0  &  1.0  &  1.0  &  Romanian  &  1.0  &  1.0  &  1.0  \\
Dutch  &  1.0  &  1.0  &  1.0  &  Russian  &  1.0  &  1.0  &  1.0  \\
English  &  1.0  &  1.0  &  1.0  &  Serbian  &  1.0  &  1.0  &  1.0  \\
Erzya  &  0.99  &  0.98  &  1.0  &  Slovak  &  0.07  &  0.03  &  0.12  \\
Estonian  &  0.8  &  0.72  &  0.86  &  Slovenian  &  0.82  &  0.77  &  0.88  \\
Faroese  &  1.0  &  1.0  &  1.0  &  Spanish  &  1.0  &  1.0  &  1.0  \\
Finnish  &  1.0  &  1.0  &  1.0  &  Swedish  &  1.0  &  1.0  &  1.0  \\
French  &  1.0  &  1.0  &  1.0  &  Thai  &  1.0  &  1.0  &  1.0  \\
German  &  1.0  &  0.91  &  1.0  &  Turkish  &  1.0  &  1.0  &  1.0  \\
Greek  &  1.0  &  1.0  &  1.0  &  Ukrainian  &  1.0  &  1.0  &  1.0  \\
Hebrew  &  1.0  &  1.0  &  1.0  &  Urdu  &  1.0  &  1.0  &  1.0  \\
Hindi  &  1.0  &  1.0  &  1.0  &  Uyghur  &  0.65  &  0.57  &  0.73  \\
Hungarian  &  0.87  &  0.8  &  0.93  &  Vietnamese  &  1.0  &  0.98  &  1.0  \\

\end{longtable}
	\caption{Bootstrapped estimates for $G$.}\label{tab:boot-g}
\end{table}



\begin{table}[!htbp]
\begin{longtable}{ccccccccccccccclll}
\input{tables/medians_REAL_0.tex}
\end{longtable}
	\caption{Medians: For each memory budget, we provide the median surprisal for real and random languages. Solid lines indicate sample medians, dashed lines indicate 95 $\%$ confidence intervals for the population median, dotted lines indicate empirical quantiles ($10\%, 20\%, \dots, 80\%, 90\%$). Green: Random baselines; blue: real language; red: maximum-likelihood grammars fit to real orderings.}\label{tab:medians}
\end{table}

\begin{table}[!htbp]
\begin{longtable}{ccccccccccccccclll}
\input{tables/medians_REAL_1.tex}
\end{longtable}
	\caption{Medians (cont.)}
\end{table}

\begin{table}[!htbp]
\begin{longtable}{ccccccccccccccclll}
\input{tables/medians_REAL_2.tex}
\end{longtable}
	\caption{Medians (cont.)}
\end{table}

\begin{table}[!htbp]
\begin{longtable}{ccccccccccccccclll}
\input{tables/medians_REAL_3.tex}
\end{longtable}
	\caption{Medians (cont.)}
\end{table}







\begin{table}[!htbp]
\begin{tabular}{ccccccccccccccclll}
\input{tables/slice-hists_REAL_0.tex}
\end{tabular}
	\caption{Histograms: Surprisal, at maximum memory.}\label{tab:slice-hists-real}
\end{table}

\begin{table}[!htbp]
\begin{tabular}{ccccccccccccccclll}
\input{tables/slice-hists_REAL_1.tex}
\end{tabular}
	\caption{Medians (cont.)}
\end{table}

\begin{table}[!htbp]
\begin{tabular}{ccccccccccccccclll}
\input{tables/slice-hists_REAL_2.tex}
\end{tabular}
	\caption{Medians (cont.)}
\end{table}

\begin{table}[!htbp]
\begin{tabular}{ccccccccccccccclll}
Thai & Turkish & Ukrainian & Urdu
 \\ 
\includegraphics[width=0.25\textwidth]{neural/figures/Thai-Adap-listener-surprisal-memory-HIST_byMem_onlyWordForms_boundedVocab_REAL.pdf} & \includegraphics[width=0.25\textwidth]{neural/figures/Turkish-listener-surprisal-memory-HIST_byMem_onlyWordForms_boundedVocab_REAL.pdf} & \includegraphics[width=0.25\textwidth]{neural/figures/Ukrainian-listener-surprisal-memory-HIST_byMem_onlyWordForms_boundedVocab_REAL.pdf} & \includegraphics[width=0.25\textwidth]{neural/figures/Urdu-listener-surprisal-memory-HIST_byMem_onlyWordForms_boundedVocab_REAL.pdf}
 \\ 
Uyghur & Vietnamese &  & 
 \\ 
\includegraphics[width=0.25\textwidth]{neural/figures/Uyghur-Adap-listener-surprisal-memory-HIST_byMem_onlyWordForms_boundedVocab_REAL.pdf} & \includegraphics[width=0.25\textwidth]{neural/figures/Vietnamese-listener-surprisal-memory-HIST_byMem_onlyWordForms_boundedVocab_REAL.pdf} &  & 
 \\ 

\end{tabular}
	\caption{Medians (cont.)}
\end{table}


%\begin{table}
%\begin{tabular}{cccccccccccccccccc}
%\input{tables/quantiles_REAL_0.tex}
%\end{tabular}
%	\caption{Quantiles: At a given memory budget, what percentage of the baselines results in higher listener surprisal than the real language? Solid curves represent sample means, dashed lines represent 95 \% confidence bounds; dotted lines represent 99.9 \% confidence bounds. At five evenly spaced memory levels, we provide a p-value for the null hypothesis that the actual population mean is $0.5$ or less. Confidence bounds and p-values are obtained using an exact nonparametric method (see text).}\label{tab:quantiles}
%\end{table}
%
%\begin{table}
%\begin{tabular}{cccccccccccccccccc}
%Finnish & French & German & Greek
 \\ 
\includegraphics[width=0.25\textwidth]{neural/figures/Finnish-listener-surprisal-memory-QUANTILES_onlyWordForms_boundedVocab_REAL.pdf} & \includegraphics[width=0.25\textwidth]{neural/figures/French-listener-surprisal-memory-QUANTILES_onlyWordForms_boundedVocab_REAL.pdf} & \includegraphics[width=0.25\textwidth]{neural/figures/German-listener-surprisal-memory-QUANTILES_onlyWordForms_boundedVocab_REAL.pdf} & \includegraphics[width=0.25\textwidth]{neural/figures/Greek-listener-surprisal-memory-QUANTILES_onlyWordForms_boundedVocab_REAL.pdf}
 \\ 
Hebrew & Hindi & Hungarian & Indonesian
 \\ 
\includegraphics[width=0.25\textwidth]{neural/figures/Hebrew-listener-surprisal-memory-QUANTILES_onlyWordForms_boundedVocab_REAL.pdf} & \includegraphics[width=0.25\textwidth]{neural/figures/Hindi-listener-surprisal-memory-QUANTILES_onlyWordForms_boundedVocab_REAL.pdf} & \includegraphics[width=0.25\textwidth]{neural/figures/Hungarian-listener-surprisal-memory-QUANTILES_onlyWordForms_boundedVocab_REAL.pdf} & \includegraphics[width=0.25\textwidth]{neural/figures/Indonesian-listener-surprisal-memory-QUANTILES_onlyWordForms_boundedVocab_REAL.pdf}
 \\ 
Italian & Japanese & Kazakh & Korean
 \\ 
\includegraphics[width=0.25\textwidth]{neural/figures/Italian-listener-surprisal-memory-QUANTILES_onlyWordForms_boundedVocab_REAL.pdf} & \includegraphics[width=0.25\textwidth]{neural/figures/Japanese-listener-surprisal-memory-QUANTILES_onlyWordForms_boundedVocab_REAL.pdf} & \includegraphics[width=0.25\textwidth]{neural/figures/Kazakh-Adap-listener-surprisal-memory-QUANTILES_onlyWordForms_boundedVocab_REAL.pdf} & \includegraphics[width=0.25\textwidth]{neural/figures/Korean-listener-surprisal-memory-QUANTILES_onlyWordForms_boundedVocab_REAL.pdf}
 \\ 
Kurmanji & Latvian & Maltese & Naija
 \\ 
\includegraphics[width=0.25\textwidth]{neural/figures/Kurmanji-Adap-listener-surprisal-memory-QUANTILES_onlyWordForms_boundedVocab_REAL.pdf} & \includegraphics[width=0.25\textwidth]{neural/figures/Latvian-listener-surprisal-memory-QUANTILES_onlyWordForms_boundedVocab_REAL.pdf} & \includegraphics[width=0.25\textwidth]{neural/figures/Maltese-listener-surprisal-memory-QUANTILES_onlyWordForms_boundedVocab_REAL.pdf} & \includegraphics[width=0.25\textwidth]{neural/figures/Naija-Adap-listener-surprisal-memory-QUANTILES_onlyWordForms_boundedVocab_REAL.pdf}
 \\ 
North Sami & Norwegian & Persian & Polish
 \\ 
\includegraphics[width=0.25\textwidth]{neural/figures/North_Sami-listener-surprisal-memory-QUANTILES_onlyWordForms_boundedVocab_REAL.pdf} & \includegraphics[width=0.25\textwidth]{neural/figures/Norwegian-listener-surprisal-memory-QUANTILES_onlyWordForms_boundedVocab_REAL.pdf} & \includegraphics[width=0.25\textwidth]{neural/figures/Persian-listener-surprisal-memory-QUANTILES_onlyWordForms_boundedVocab_REAL.pdf} & \includegraphics[width=0.25\textwidth]{neural/figures/Polish-listener-surprisal-memory-QUANTILES_onlyWordForms_boundedVocab_REAL.pdf}
 \\ 

%\end{tabular}
%	\caption{Quantiles (part 2)}
%\end{table}
%
%\begin{table}
%\begin{tabular}{cccccccccccccccccc}
%\input{tables/quantiles_REAL_2.tex}
%\end{tabular}
%	\caption{Quantiles (part 3)}
%\end{table}
%
%
%


















\begin{table}[!htbp]
\begin{tabular}{l|ll||l|llllllllllllll}
	Language & Base. & MLE & Language & Base. & MLE \\ \hline
Afrikaans  &  13  &  10  &  Indonesian  &  11  &  10  \\
Amharic  &  137  &  71  &  Italian  &  10  &  10  \\
Arabic  &  11  &  10  &  Japanese  &  25  &  10  \\
Armenian  &  140  &  17  &  Kazakh  &  11  &  10  \\
Bambara  &  25  &  10  &  Korean  &  11  &  10  \\
Basque  &  15  &  10  &  Kurmanji  &  338  &  101  \\
Breton  &  35  &  10  &  Latvian  &  308  &  132  \\
Bulgarian  &  14  &  10  &  Maltese  &  30  &  10  \\
Buryat  &  26  &  10  &  Naija  &  214  &  93  \\
Cantonese  &  306  &  135  &  North Sami  &  335  &  101  \\
Catalan  &  11  &  10  &  Norwegian  &  12  &  10  \\
Chinese  &  21  &  10  &  Persian  &  25  &  10  \\
Croatian  &  30  &  10  &  Polish  &  309  &  131  \\
Czech  &  18  &  12  &  Portuguese  &  15  &  99  \\
Danish  &  33  &  10  &  Romanian  &  10  &  10  \\
Dutch  &  27  &  10  &  Russian  &  20  &  13  \\
English  &  13  &  10  &  Serbian  &  26  &  11  \\
Erzya  &  846  &  101  &  Slovak  &  303  &  138  \\
Estonian  &  347  &  10  &  Slovenian  &  297  &  12  \\
Faroese  &  27  &  10  &  Spanish  &  14  &  10  \\
Finnish  &  83  &  54  &  Swedish  &  31  &  10  \\
French  &  14  &  12  &  Thai  &  45  &  10  \\
German  &  19  &  10  &  Turkish  &  13  &  10  \\
Greek  &  16  &  10  &  Ukrainian  &  28  &  10  \\
Hebrew  &  11  &  10  &  Urdu  &  17  &  10  \\
Hindi  &  11  &  10  &  Uyghur  &  326  &  132  \\
Hungarian  &  220  &  35  &  Vietnamese  &  303  &  132  \\

\end{tabular}
	\caption{Experiment 3: Samples drawn per language according to the precision-dependent stopping criterion.}\label{tab:samples}
\end{table}





\begin{table}[!htbp]
\begin{tabular}{ccccccccccccccclll}
Afrikaans & Amharic & Arabic & Armenian
 \\ 
\includegraphics[width=0.25\textwidth]{neural/figures/Afrikaans-listener-surprisal-memory-MEDIANS_QUANTILES_onlyWordForms_boundedVocab.pdf} & \includegraphics[width=0.25\textwidth]{neural/figures/Amharic-Adap-listener-surprisal-memory-MEDIANS_QUANTILES_onlyWordForms_boundedVocab.pdf} & \includegraphics[width=0.25\textwidth]{neural/figures/Arabic-listener-surprisal-memory-MEDIANS_QUANTILES_onlyWordForms_boundedVocab.pdf} & \includegraphics[width=0.25\textwidth]{neural/figures/Armenian-Adap-listener-surprisal-memory-MEDIANS_QUANTILES_onlyWordForms_boundedVocab.pdf}
 \\ 
Bambara & Basque & Breton & Bulgarian
 \\ 
\includegraphics[width=0.25\textwidth]{neural/figures/Bambara-Adap-listener-surprisal-memory-MEDIANS_QUANTILES_onlyWordForms_boundedVocab.pdf} & \includegraphics[width=0.25\textwidth]{neural/figures/Basque-listener-surprisal-memory-MEDIANS_QUANTILES_onlyWordForms_boundedVocab.pdf} & \includegraphics[width=0.25\textwidth]{neural/figures/Breton-Adap-listener-surprisal-memory-MEDIANS_QUANTILES_onlyWordForms_boundedVocab.pdf} & \includegraphics[width=0.25\textwidth]{neural/figures/Bulgarian-listener-surprisal-memory-MEDIANS_QUANTILES_onlyWordForms_boundedVocab.pdf}
 \\ 
Buryat & Cantonese & Catalan & Chinese
 \\ 
\includegraphics[width=0.25\textwidth]{neural/figures/Buryat-Adap-listener-surprisal-memory-MEDIANS_QUANTILES_onlyWordForms_boundedVocab.pdf} & \includegraphics[width=0.25\textwidth]{neural/figures/Cantonese-Adap-listener-surprisal-memory-MEDIANS_QUANTILES_onlyWordForms_boundedVocab.pdf} & \includegraphics[width=0.25\textwidth]{neural/figures/Catalan-listener-surprisal-memory-MEDIANS_QUANTILES_onlyWordForms_boundedVocab.pdf} & \includegraphics[width=0.25\textwidth]{neural/figures/Chinese-listener-surprisal-memory-MEDIANS_QUANTILES_onlyWordForms_boundedVocab.pdf}
 \\ 
Croatian & Czech & Danish & Dutch
 \\ 
\includegraphics[width=0.25\textwidth]{neural/figures/Croatian-listener-surprisal-memory-MEDIANS_QUANTILES_onlyWordForms_boundedVocab.pdf} & \includegraphics[width=0.25\textwidth]{neural/figures/Czech-listener-surprisal-memory-MEDIANS_QUANTILES_onlyWordForms_boundedVocab.pdf} & \includegraphics[width=0.25\textwidth]{neural/figures/Danish-listener-surprisal-memory-MEDIANS_QUANTILES_onlyWordForms_boundedVocab.pdf} & \includegraphics[width=0.25\textwidth]{neural/figures/Dutch-listener-surprisal-memory-MEDIANS_QUANTILES_onlyWordForms_boundedVocab.pdf}
 \\ 

\end{tabular}
	\caption{Experiment 3. Medians: For each memory budget, we provide the median surprisal for real and random languages. Solid lines indicate sample medians, dashed lines indicate 95 $\%$ confidence intervals for the population median. Green: Random baselines; blue: real language; red: maximum-likelihood grammars fit to real orderings.}\label{tab:medians}
\end{table}

\begin{table}[!htbp]
\begin{tabular}{ccccccccccccccclll}
English & Erzya & Estonian & Faroese
 \\ 
\includegraphics[width=0.25\textwidth]{neural/figures/English-listener-surprisal-memory-MEDIANS_QUANTILES_onlyWordForms_boundedVocab.pdf} & \includegraphics[width=0.25\textwidth]{neural/figures/Erzya-Adap-listener-surprisal-memory-MEDIANS_QUANTILES_onlyWordForms_boundedVocab.pdf} & \includegraphics[width=0.25\textwidth]{neural/figures/Estonian-listener-surprisal-memory-MEDIANS_QUANTILES_onlyWordForms_boundedVocab.pdf} & \includegraphics[width=0.25\textwidth]{neural/figures/Faroese-Adap-listener-surprisal-memory-MEDIANS_QUANTILES_onlyWordForms_boundedVocab.pdf}
 \\ 
Finnish & French & German & Greek
 \\ 
\includegraphics[width=0.25\textwidth]{neural/figures/Finnish-listener-surprisal-memory-MEDIANS_QUANTILES_onlyWordForms_boundedVocab.pdf} & \includegraphics[width=0.25\textwidth]{neural/figures/French-listener-surprisal-memory-MEDIANS_QUANTILES_onlyWordForms_boundedVocab.pdf} & \includegraphics[width=0.25\textwidth]{neural/figures/German-listener-surprisal-memory-MEDIANS_QUANTILES_onlyWordForms_boundedVocab.pdf} & \includegraphics[width=0.25\textwidth]{neural/figures/Greek-listener-surprisal-memory-MEDIANS_QUANTILES_onlyWordForms_boundedVocab.pdf}
 \\ 
Hebrew & Hindi & Hungarian & Indonesian
 \\ 
\includegraphics[width=0.25\textwidth]{neural/figures/Hebrew-listener-surprisal-memory-MEDIANS_QUANTILES_onlyWordForms_boundedVocab.pdf} & \includegraphics[width=0.25\textwidth]{neural/figures/Hindi-listener-surprisal-memory-MEDIANS_QUANTILES_onlyWordForms_boundedVocab.pdf} & \includegraphics[width=0.25\textwidth]{neural/figures/Hungarian-listener-surprisal-memory-MEDIANS_QUANTILES_onlyWordForms_boundedVocab.pdf} & \includegraphics[width=0.25\textwidth]{neural/figures/Indonesian-listener-surprisal-memory-MEDIANS_QUANTILES_onlyWordForms_boundedVocab.pdf}
 \\ 
Italian & Japanese & Kazakh & Korean
 \\ 
\includegraphics[width=0.25\textwidth]{neural/figures/Italian-listener-surprisal-memory-MEDIANS_QUANTILES_onlyWordForms_boundedVocab.pdf} & \includegraphics[width=0.25\textwidth]{neural/figures/Japanese-listener-surprisal-memory-MEDIANS_QUANTILES_onlyWordForms_boundedVocab.pdf} & \includegraphics[width=0.25\textwidth]{neural/figures/Kazakh-Adap-listener-surprisal-memory-MEDIANS_QUANTILES_onlyWordForms_boundedVocab.pdf} & \includegraphics[width=0.25\textwidth]{neural/figures/Korean-listener-surprisal-memory-MEDIANS_QUANTILES_onlyWordForms_boundedVocab.pdf}
 \\ 

\end{tabular}
	\caption{Medians (cont.)}
\end{table}

\begin{table}[!htbp]
\begin{tabular}{ccccccccccccccclll}
Kurmanji & Latvian & Maltese & Naija
 \\ 
\includegraphics[width=0.25\textwidth]{neural/figures/Kurmanji-Adap-listener-surprisal-memory-MEDIANS_QUANTILES_onlyWordForms_boundedVocab.pdf} & \includegraphics[width=0.25\textwidth]{neural/figures/Latvian-listener-surprisal-memory-MEDIANS_QUANTILES_onlyWordForms_boundedVocab.pdf} & \includegraphics[width=0.25\textwidth]{neural/figures/Maltese-listener-surprisal-memory-MEDIANS_QUANTILES_onlyWordForms_boundedVocab.pdf} & \includegraphics[width=0.25\textwidth]{neural/figures/Naija-Adap-listener-surprisal-memory-MEDIANS_QUANTILES_onlyWordForms_boundedVocab.pdf}
 \\ 
North Sami & Norwegian & Persian & Polish
 \\ 
\includegraphics[width=0.25\textwidth]{neural/figures/North_Sami-listener-surprisal-memory-MEDIANS_QUANTILES_onlyWordForms_boundedVocab.pdf} & \includegraphics[width=0.25\textwidth]{neural/figures/Norwegian-listener-surprisal-memory-MEDIANS_QUANTILES_onlyWordForms_boundedVocab.pdf} & \includegraphics[width=0.25\textwidth]{neural/figures/Persian-listener-surprisal-memory-MEDIANS_QUANTILES_onlyWordForms_boundedVocab.pdf} & \includegraphics[width=0.25\textwidth]{neural/figures/Polish-listener-surprisal-memory-MEDIANS_QUANTILES_onlyWordForms_boundedVocab.pdf}
 \\ 
Portuguese & Romanian & Russian & Serbian
 \\ 
\includegraphics[width=0.25\textwidth]{neural/figures/Portuguese-listener-surprisal-memory-MEDIANS_QUANTILES_onlyWordForms_boundedVocab.pdf} & \includegraphics[width=0.25\textwidth]{neural/figures/Romanian-listener-surprisal-memory-MEDIANS_QUANTILES_onlyWordForms_boundedVocab.pdf} & \includegraphics[width=0.25\textwidth]{neural/figures/Russian-listener-surprisal-memory-MEDIANS_QUANTILES_onlyWordForms_boundedVocab.pdf} & \includegraphics[width=0.25\textwidth]{neural/figures/Serbian-listener-surprisal-memory-MEDIANS_QUANTILES_onlyWordForms_boundedVocab.pdf}
 \\ 
Slovak & Slovenian & Spanish & Swedish
 \\ 
\includegraphics[width=0.25\textwidth]{neural/figures/Slovak-listener-surprisal-memory-MEDIANS_QUANTILES_onlyWordForms_boundedVocab.pdf} & \includegraphics[width=0.25\textwidth]{neural/figures/Slovenian-listener-surprisal-memory-MEDIANS_QUANTILES_onlyWordForms_boundedVocab.pdf} & \includegraphics[width=0.25\textwidth]{neural/figures/Spanish-listener-surprisal-memory-MEDIANS_QUANTILES_onlyWordForms_boundedVocab.pdf} & \includegraphics[width=0.25\textwidth]{neural/figures/Swedish-listener-surprisal-memory-MEDIANS_QUANTILES_onlyWordForms_boundedVocab.pdf}
 \\ 

\end{tabular}
	\caption{Medians (cont.)}
\end{table}

\begin{table}[!htbp]
\begin{tabular}{ccccccccccccccclll}
\input{tables/medians_3.tex}
\end{tabular}
	\caption{Medians (cont.)}
\end{table}




\begin{table}[!htbp]
\begin{tabular}{ccccccccccccccccll}
Afrikaans & Amharic & Arabic & Armenian
 \\ 
\includegraphics[width=0.25\textwidth]{neural/figures/Afrikaans-listener-surprisal-memory-MEDIAN_DIFFS_onlyWordForms_boundedVocab.pdf} & \includegraphics[width=0.25\textwidth]{neural/figures/Amharic-Adap-listener-surprisal-memory-MEDIAN_DIFFS_onlyWordForms_boundedVocab.pdf} & \includegraphics[width=0.25\textwidth]{neural/figures/Arabic-listener-surprisal-memory-MEDIAN_DIFFS_onlyWordForms_boundedVocab.pdf} & \includegraphics[width=0.25\textwidth]{neural/figures/Armenian-Adap-listener-surprisal-memory-MEDIAN_DIFFS_onlyWordForms_boundedVocab.pdf}
 \\ 
Bambara & Basque & Breton & Bulgarian
 \\ 
\includegraphics[width=0.25\textwidth]{neural/figures/Bambara-Adap-listener-surprisal-memory-MEDIAN_DIFFS_onlyWordForms_boundedVocab.pdf} & \includegraphics[width=0.25\textwidth]{neural/figures/Basque-listener-surprisal-memory-MEDIAN_DIFFS_onlyWordForms_boundedVocab.pdf} & \includegraphics[width=0.25\textwidth]{neural/figures/Breton-Adap-listener-surprisal-memory-MEDIAN_DIFFS_onlyWordForms_boundedVocab.pdf} & \includegraphics[width=0.25\textwidth]{neural/figures/Bulgarian-listener-surprisal-memory-MEDIAN_DIFFS_onlyWordForms_boundedVocab.pdf}
 \\ 
Buryat & Cantonese & Catalan & Chinese
 \\ 
\includegraphics[width=0.25\textwidth]{neural/figures/Buryat-Adap-listener-surprisal-memory-MEDIAN_DIFFS_onlyWordForms_boundedVocab.pdf} & \includegraphics[width=0.25\textwidth]{neural/figures/Cantonese-Adap-listener-surprisal-memory-MEDIAN_DIFFS_onlyWordForms_boundedVocab.pdf} & \includegraphics[width=0.25\textwidth]{neural/figures/Catalan-listener-surprisal-memory-MEDIAN_DIFFS_onlyWordForms_boundedVocab.pdf} & \includegraphics[width=0.25\textwidth]{neural/figures/Chinese-listener-surprisal-memory-MEDIAN_DIFFS_onlyWordForms_boundedVocab.pdf}
 \\ 
Croatian & Czech & Danish & Dutch
 \\ 
\includegraphics[width=0.25\textwidth]{neural/figures/Croatian-listener-surprisal-memory-MEDIAN_DIFFS_onlyWordForms_boundedVocab.pdf} & \includegraphics[width=0.25\textwidth]{neural/figures/Czech-listener-surprisal-memory-MEDIAN_DIFFS_onlyWordForms_boundedVocab.pdf} & \includegraphics[width=0.25\textwidth]{neural/figures/Danish-listener-surprisal-memory-MEDIAN_DIFFS_onlyWordForms_boundedVocab.pdf} & \includegraphics[width=0.25\textwidth]{neural/figures/Dutch-listener-surprisal-memory-MEDIAN_DIFFS_onlyWordForms_boundedVocab.pdf}
 \\ 

\end{tabular}
	\caption{Median Differences between Real and Baseline: For each memory budget, we provide the difference in median surprisal between real languages and random baselines; for real orders (blue) and maximum likelihood grammars (red). Lower values indicate lower surprisal compared to baselines. Solid lines indicate sample means. Dashed lines indicate 95 $\%$ confidence intervals.}\label{tab:median_diffs}
\end{table}

\begin{table}[!htbp]
\begin{tabular}{ccccccccccccccccll}
\input{tables/medianDiff_1.tex}
\end{tabular}
	\caption{Median Differences (Part 2)}
\end{table}

\begin{table}[!htbp]
\begin{tabular}{ccccccccccccccccll}
Kurmanji & Latvian & Maltese & Naija
 \\ 
\includegraphics[width=0.25\textwidth]{neural/figures/Kurmanji-Adap-listener-surprisal-memory-MEDIAN_DIFFS_onlyWordForms_boundedVocab.pdf} & \includegraphics[width=0.25\textwidth]{neural/figures/Latvian-listener-surprisal-memory-MEDIAN_DIFFS_onlyWordForms_boundedVocab.pdf} & \includegraphics[width=0.25\textwidth]{neural/figures/Maltese-listener-surprisal-memory-MEDIAN_DIFFS_onlyWordForms_boundedVocab.pdf} & \includegraphics[width=0.25\textwidth]{neural/figures/Naija-Adap-listener-surprisal-memory-MEDIAN_DIFFS_onlyWordForms_boundedVocab.pdf}
 \\ 
North Sami & Norwegian & Persian & Polish
 \\ 
\includegraphics[width=0.25\textwidth]{neural/figures/North_Sami-listener-surprisal-memory-MEDIAN_DIFFS_onlyWordForms_boundedVocab.pdf} & \includegraphics[width=0.25\textwidth]{neural/figures/Norwegian-listener-surprisal-memory-MEDIAN_DIFFS_onlyWordForms_boundedVocab.pdf} & \includegraphics[width=0.25\textwidth]{neural/figures/Persian-listener-surprisal-memory-MEDIAN_DIFFS_onlyWordForms_boundedVocab.pdf} & \includegraphics[width=0.25\textwidth]{neural/figures/Polish-listener-surprisal-memory-MEDIAN_DIFFS_onlyWordForms_boundedVocab.pdf}
 \\ 
Portuguese & Romanian & Russian & Serbian
 \\ 
\includegraphics[width=0.25\textwidth]{neural/figures/Portuguese-listener-surprisal-memory-MEDIAN_DIFFS_onlyWordForms_boundedVocab.pdf} & \includegraphics[width=0.25\textwidth]{neural/figures/Romanian-listener-surprisal-memory-MEDIAN_DIFFS_onlyWordForms_boundedVocab.pdf} & \includegraphics[width=0.25\textwidth]{neural/figures/Russian-listener-surprisal-memory-MEDIAN_DIFFS_onlyWordForms_boundedVocab.pdf} & \includegraphics[width=0.25\textwidth]{neural/figures/Serbian-listener-surprisal-memory-MEDIAN_DIFFS_onlyWordForms_boundedVocab.pdf}
 \\ 
Slovak & Slovenian & Spanish & Swedish
 \\ 
\includegraphics[width=0.25\textwidth]{neural/figures/Slovak-listener-surprisal-memory-MEDIAN_DIFFS_onlyWordForms_boundedVocab.pdf} & \includegraphics[width=0.25\textwidth]{neural/figures/Slovenian-listener-surprisal-memory-MEDIAN_DIFFS_onlyWordForms_boundedVocab.pdf} & \includegraphics[width=0.25\textwidth]{neural/figures/Spanish-listener-surprisal-memory-MEDIAN_DIFFS_onlyWordForms_boundedVocab.pdf} & \includegraphics[width=0.25\textwidth]{neural/figures/Swedish-listener-surprisal-memory-MEDIAN_DIFFS_onlyWordForms_boundedVocab.pdf}
 \\ 

\end{tabular}
	\caption{Median Differences (Part 3)}
\end{table}

\begin{table}[!htbp]
\begin{tabular}{ccccccccccccccccll}
Thai & Turkish & Ukrainian & Urdu
 \\ 
\includegraphics[width=0.25\textwidth]{neural/figures/Thai-Adap-listener-surprisal-memory-MEDIAN_DIFFS_onlyWordForms_boundedVocab.pdf} & \includegraphics[width=0.25\textwidth]{neural/figures/Turkish-listener-surprisal-memory-MEDIAN_DIFFS_onlyWordForms_boundedVocab.pdf} & \includegraphics[width=0.25\textwidth]{neural/figures/Ukrainian-listener-surprisal-memory-MEDIAN_DIFFS_onlyWordForms_boundedVocab.pdf} & \includegraphics[width=0.25\textwidth]{neural/figures/Urdu-listener-surprisal-memory-MEDIAN_DIFFS_onlyWordForms_boundedVocab.pdf}
 \\ 
Uyghur & Vietnamese &  & 
 \\ 
\includegraphics[width=0.25\textwidth]{neural/figures/Uyghur-Adap-listener-surprisal-memory-MEDIAN_DIFFS_onlyWordForms_boundedVocab.pdf} & \includegraphics[width=0.25\textwidth]{neural/figures/Vietnamese-listener-surprisal-memory-MEDIAN_DIFFS_onlyWordForms_boundedVocab.pdf} &  & 
 \\ 

\end{tabular}
	\caption{Median Differences (Part 4)}
\end{table}






\begin{table}[!htbp]
\begin{tabular}{cccccccccccccccccc}
Afrikaans & Amharic & Arabic & Armenian
 \\ 
\includegraphics[width=0.25\textwidth]{neural/figures/Afrikaans-listener-surprisal-memory-QUANTILES_onlyWordForms_boundedVocab_noAssumption.pdf} & \includegraphics[width=0.25\textwidth]{neural/figures/Amharic-Adap-listener-surprisal-memory-QUANTILES_onlyWordForms_boundedVocab_noAssumption.pdf} & \includegraphics[width=0.25\textwidth]{neural/figures/Arabic-listener-surprisal-memory-QUANTILES_onlyWordForms_boundedVocab_noAssumption.pdf} & \includegraphics[width=0.25\textwidth]{neural/figures/Armenian-Adap-listener-surprisal-memory-QUANTILES_onlyWordForms_boundedVocab_noAssumption.pdf}
 \\ 
Bambara & Basque & Breton & Bulgarian
 \\ 
\includegraphics[width=0.25\textwidth]{neural/figures/Bambara-Adap-listener-surprisal-memory-QUANTILES_onlyWordForms_boundedVocab_noAssumption.pdf} & \includegraphics[width=0.25\textwidth]{neural/figures/Basque-listener-surprisal-memory-QUANTILES_onlyWordForms_boundedVocab_noAssumption.pdf} & \includegraphics[width=0.25\textwidth]{neural/figures/Breton-Adap-listener-surprisal-memory-QUANTILES_onlyWordForms_boundedVocab_noAssumption.pdf} & \includegraphics[width=0.25\textwidth]{neural/figures/Bulgarian-listener-surprisal-memory-QUANTILES_onlyWordForms_boundedVocab_noAssumption.pdf}
 \\ 
Buryat & Cantonese & Catalan & Chinese
 \\ 
\includegraphics[width=0.25\textwidth]{neural/figures/Buryat-Adap-listener-surprisal-memory-QUANTILES_onlyWordForms_boundedVocab_noAssumption.pdf} & \includegraphics[width=0.25\textwidth]{neural/figures/Cantonese-Adap-listener-surprisal-memory-QUANTILES_onlyWordForms_boundedVocab_noAssumption.pdf} & \includegraphics[width=0.25\textwidth]{neural/figures/Catalan-listener-surprisal-memory-QUANTILES_onlyWordForms_boundedVocab_noAssumption.pdf} & \includegraphics[width=0.25\textwidth]{neural/figures/Chinese-listener-surprisal-memory-QUANTILES_onlyWordForms_boundedVocab_noAssumption.pdf}
 \\ 
Croatian & Czech & Danish & Dutch
 \\ 
\includegraphics[width=0.25\textwidth]{neural/figures/Croatian-listener-surprisal-memory-QUANTILES_onlyWordForms_boundedVocab_noAssumption.pdf} & \includegraphics[width=0.25\textwidth]{neural/figures/Czech-listener-surprisal-memory-QUANTILES_onlyWordForms_boundedVocab_noAssumption.pdf} & \includegraphics[width=0.25\textwidth]{neural/figures/Danish-listener-surprisal-memory-QUANTILES_onlyWordForms_boundedVocab_noAssumption.pdf} & \includegraphics[width=0.25\textwidth]{neural/figures/Dutch-listener-surprisal-memory-QUANTILES_onlyWordForms_boundedVocab_noAssumption.pdf}
 \\ 
English & Erzya & Estonian & Faroese
 \\ 
\includegraphics[width=0.25\textwidth]{neural/figures/English-listener-surprisal-memory-QUANTILES_onlyWordForms_boundedVocab_noAssumption.pdf} & \includegraphics[width=0.25\textwidth]{neural/figures/Erzya-Adap-listener-surprisal-memory-QUANTILES_onlyWordForms_boundedVocab_noAssumption.pdf} & \includegraphics[width=0.25\textwidth]{neural/figures/Estonian-listener-surprisal-memory-QUANTILES_onlyWordForms_boundedVocab_noAssumption.pdf} & \includegraphics[width=0.25\textwidth]{neural/figures/Faroese-Adap-listener-surprisal-memory-QUANTILES_onlyWordForms_boundedVocab_noAssumption.pdf}
 \\ 

\end{tabular}
	\caption{Quantiles: At a given memory budget, what percentage of the baselines results in higher listener surprisal than the real language? Solid curves represent sample means, dashed lines represent 95 \% confidence bounds; dotted lines represent 99.9 \% confidence bounds. At five evenly spaced memory levels, we provide a p-value for the null hypothesis that the actual population mean is $0.5$ or less. Confidence bounds and p-values are obtained using an exact nonparametric method (see text).}\label{tab:quantiles}
\end{table}

\begin{table}[!htbp]
\begin{tabular}{cccccccccccccccccc}
\input{tables/quantiles_noAssumption_1.tex}
\end{tabular}
	\caption{Quantiles (part 2)}
\end{table}

\begin{table}[!htbp]
\begin{tabular}{cccccccccccccccccc}
Portuguese & Romanian & Russian & Serbian
 \\ 
\includegraphics[width=0.25\textwidth]{neural/figures/Portuguese-listener-surprisal-memory-QUANTILES_onlyWordForms_boundedVocab_noAssumption.pdf} & \includegraphics[width=0.25\textwidth]{neural/figures/Romanian-listener-surprisal-memory-QUANTILES_onlyWordForms_boundedVocab_noAssumption.pdf} & \includegraphics[width=0.25\textwidth]{neural/figures/Russian-listener-surprisal-memory-QUANTILES_onlyWordForms_boundedVocab_noAssumption.pdf} & \includegraphics[width=0.25\textwidth]{neural/figures/Serbian-listener-surprisal-memory-QUANTILES_onlyWordForms_boundedVocab_noAssumption.pdf}
 \\ 
Slovak & Slovenian & Spanish & Swedish
 \\ 
\includegraphics[width=0.25\textwidth]{neural/figures/Slovak-listener-surprisal-memory-QUANTILES_onlyWordForms_boundedVocab_noAssumption.pdf} & \includegraphics[width=0.25\textwidth]{neural/figures/Slovenian-listener-surprisal-memory-QUANTILES_onlyWordForms_boundedVocab_noAssumption.pdf} & \includegraphics[width=0.25\textwidth]{neural/figures/Spanish-listener-surprisal-memory-QUANTILES_onlyWordForms_boundedVocab_noAssumption.pdf} & \includegraphics[width=0.25\textwidth]{neural/figures/Swedish-listener-surprisal-memory-QUANTILES_onlyWordForms_boundedVocab_noAssumption.pdf}
 \\ 
Thai & Turkish & Ukrainian & Urdu
 \\ 
\includegraphics[width=0.25\textwidth]{neural/figures/Thai-Adap-listener-surprisal-memory-QUANTILES_onlyWordForms_boundedVocab_noAssumption.pdf} & \includegraphics[width=0.25\textwidth]{neural/figures/Turkish-listener-surprisal-memory-QUANTILES_onlyWordForms_boundedVocab_noAssumption.pdf} & \includegraphics[width=0.25\textwidth]{neural/figures/Ukrainian-listener-surprisal-memory-QUANTILES_onlyWordForms_boundedVocab_noAssumption.pdf} & \includegraphics[width=0.25\textwidth]{neural/figures/Urdu-listener-surprisal-memory-QUANTILES_onlyWordForms_boundedVocab_noAssumption.pdf}
 \\ 
Uyghur & Vietnamese &  & 
 \\ 
\includegraphics[width=0.25\textwidth]{neural/figures/Uyghur-Adap-listener-surprisal-memory-QUANTILES_onlyWordForms_boundedVocab_noAssumption.pdf} & \includegraphics[width=0.25\textwidth]{neural/figures/Vietnamese-listener-surprisal-memory-QUANTILES_onlyWordForms_boundedVocab_noAssumption.pdf} &  & 
 \\ 

\end{tabular}
	\caption{Quantiles (part 3)}
\end{table}







\section{N-Gram Models}


%bounds.append(["alpha", float, 0.95, 1.0]) # + [x/20.0 for x in range(15, 21)])
%bounds.append(["gamma", int, 1, 2, 3, 4, 5, 8, 10, 15, 20, 25, 30]) # , 200, 300
%bounds.append(["delta", int, 0.1, 0.2, 0.5, 1.0 , 2.0, 3.0, 4.0, 5.0, 8.0, 10.0]) #, 1024]) #, 1024]) # 64, 128,
%bounds.append(["cutoff", int, 2,3,4,5,6,7,8,9,10]) #,7,8,9,10]) #, 1024]) #, 1024]) # 64, 128,
%

We use Interpolated Kneser-Ney.
Hyperparameters are tuned with the same strategy as for the neural network models.



\begin{table}
\begin{tabular}{ccccccccccccccclll}
Afrikaans & Amharic & Arabic & Armenian
 \\ 
\includegraphics[width=0.25\textwidth]{ngrams/figures/Afrikaans-listener-surprisal-memory-MEDIANS_onlyWordForms_boundedVocab.pdf} & \includegraphics[width=0.25\textwidth]{ngrams/figures/Amharic-Adap-listener-surprisal-memory-MEDIANS_onlyWordForms_boundedVocab.pdf} & \includegraphics[width=0.25\textwidth]{ngrams/figures/Arabic-listener-surprisal-memory-MEDIANS_onlyWordForms_boundedVocab.pdf} & \includegraphics[width=0.25\textwidth]{ngrams/figures/Armenian-Adap-listener-surprisal-memory-MEDIANS_onlyWordForms_boundedVocab.pdf}
 \\ 
Basque & Breton & Bulgarian & Buryat
 \\ 
\includegraphics[width=0.25\textwidth]{ngrams/figures/Basque-listener-surprisal-memory-MEDIANS_onlyWordForms_boundedVocab.pdf} & \includegraphics[width=0.25\textwidth]{ngrams/figures/Breton-Adap-listener-surprisal-memory-MEDIANS_onlyWordForms_boundedVocab.pdf} & \includegraphics[width=0.25\textwidth]{ngrams/figures/Bulgarian-listener-surprisal-memory-MEDIANS_onlyWordForms_boundedVocab.pdf} & \includegraphics[width=0.25\textwidth]{ngrams/figures/Buryat-Adap-listener-surprisal-memory-MEDIANS_onlyWordForms_boundedVocab.pdf}
 \\ 
Cantonese & Catalan & Chinese & Croatian
 \\ 
\includegraphics[width=0.25\textwidth]{ngrams/figures/Cantonese-Adap-listener-surprisal-memory-MEDIANS_onlyWordForms_boundedVocab.pdf} & \includegraphics[width=0.25\textwidth]{ngrams/figures/Catalan-listener-surprisal-memory-MEDIANS_onlyWordForms_boundedVocab.pdf} & \includegraphics[width=0.25\textwidth]{ngrams/figures/Chinese-listener-surprisal-memory-MEDIANS_onlyWordForms_boundedVocab.pdf} & \includegraphics[width=0.25\textwidth]{ngrams/figures/Croatian-listener-surprisal-memory-MEDIANS_onlyWordForms_boundedVocab.pdf}
 \\ 
Czech & Danish & Dutch & Estonian
 \\ 
\includegraphics[width=0.25\textwidth]{ngrams/figures/Czech-listener-surprisal-memory-MEDIANS_onlyWordForms_boundedVocab.pdf} & \includegraphics[width=0.25\textwidth]{ngrams/figures/Danish-listener-surprisal-memory-MEDIANS_onlyWordForms_boundedVocab.pdf} & \includegraphics[width=0.25\textwidth]{ngrams/figures/Dutch-listener-surprisal-memory-MEDIANS_onlyWordForms_boundedVocab.pdf} & \includegraphics[width=0.25\textwidth]{ngrams/figures/Estonian-listener-surprisal-memory-MEDIANS_onlyWordForms_boundedVocab.pdf}
 \\ 

\end{tabular}
	\caption{Medians (estimated using n-gram models): For each memory budget, we provide the median surprisal for real and random languages. Solid lines indicate sample medians for ngrams, dashed lines indicate 95 \% confidence intervals for the population median. Green: Random baselines; blue: real language; red: maximum-likelihood grammars fit to real orderings.}\label{tab:medians_ngrams}
\end{table}

\begin{table}
\begin{tabular}{ccccccccccccccclll}
English & Erzya & Estonian & Faroese
 \\ 
\includegraphics[width=0.25\textwidth]{../code/analyze_ngrams/visualize/figures/English-listener-surprisal-memory-MEDIANS_onlyWordForms_boundedVocab.pdf} & \includegraphics[width=0.25\textwidth]{../code/analyze_ngrams/visualize/figures/Erzya-Adap-listener-surprisal-memory-MEDIANS_onlyWordForms_boundedVocab.pdf} & \includegraphics[width=0.25\textwidth]{../code/analyze_ngrams/visualize/figures/Estonian-listener-surprisal-memory-MEDIANS_onlyWordForms_boundedVocab.pdf} & \includegraphics[width=0.25\textwidth]{../code/analyze_ngrams/visualize/figures/Faroese-Adap-listener-surprisal-memory-MEDIANS_onlyWordForms_boundedVocab.pdf}
 \\ 
Finnish & French & German & Greek
 \\ 
\includegraphics[width=0.25\textwidth]{../code/analyze_ngrams/visualize/figures/Finnish-listener-surprisal-memory-MEDIANS_onlyWordForms_boundedVocab.pdf} & \includegraphics[width=0.25\textwidth]{../code/analyze_ngrams/visualize/figures/French-listener-surprisal-memory-MEDIANS_onlyWordForms_boundedVocab.pdf} & \includegraphics[width=0.25\textwidth]{../code/analyze_ngrams/visualize/figures/German-listener-surprisal-memory-MEDIANS_onlyWordForms_boundedVocab.pdf} & \includegraphics[width=0.25\textwidth]{../code/analyze_ngrams/visualize/figures/Greek-listener-surprisal-memory-MEDIANS_onlyWordForms_boundedVocab.pdf}
 \\ 
Hebrew & Hindi & Hungarian & Indonesian
 \\ 
\includegraphics[width=0.25\textwidth]{../code/analyze_ngrams/visualize/figures/Hebrew-listener-surprisal-memory-MEDIANS_onlyWordForms_boundedVocab.pdf} & \includegraphics[width=0.25\textwidth]{../code/analyze_ngrams/visualize/figures/Hindi-listener-surprisal-memory-MEDIANS_onlyWordForms_boundedVocab.pdf} & \includegraphics[width=0.25\textwidth]{../code/analyze_ngrams/visualize/figures/Hungarian-listener-surprisal-memory-MEDIANS_onlyWordForms_boundedVocab.pdf} & \includegraphics[width=0.25\textwidth]{../code/analyze_ngrams/visualize/figures/Indonesian-listener-surprisal-memory-MEDIANS_onlyWordForms_boundedVocab.pdf}
 \\ 
Italian & Japanese & Kazakh & Korean
 \\ 
\includegraphics[width=0.25\textwidth]{../code/analyze_ngrams/visualize/figures/Italian-listener-surprisal-memory-MEDIANS_onlyWordForms_boundedVocab.pdf} & \includegraphics[width=0.25\textwidth]{../code/analyze_ngrams/visualize/figures/Japanese-listener-surprisal-memory-MEDIANS_onlyWordForms_boundedVocab.pdf} & \includegraphics[width=0.25\textwidth]{../code/analyze_ngrams/visualize/figures/Kazakh-Adap-listener-surprisal-memory-MEDIANS_onlyWordForms_boundedVocab.pdf} & \includegraphics[width=0.25\textwidth]{../code/analyze_ngrams/visualize/figures/Korean-listener-surprisal-memory-MEDIANS_onlyWordForms_boundedVocab.pdf}
 \\ 

\end{tabular}
	\caption{Medians (cont.)}
\end{table}

\begin{table}
\begin{tabular}{ccccccccccccccclll}
Kurmanji & Latvian & Maltese & Naija
 \\ 
\includegraphics[width=0.25\textwidth]{../code/analyze_ngrams/visualize/figures/Kurmanji-Adap-listener-surprisal-memory-MEDIANS_onlyWordForms_boundedVocab.pdf} & \includegraphics[width=0.25\textwidth]{../code/analyze_ngrams/visualize/figures/Latvian-listener-surprisal-memory-MEDIANS_onlyWordForms_boundedVocab.pdf} & \includegraphics[width=0.25\textwidth]{../code/analyze_ngrams/visualize/figures/Maltese-listener-surprisal-memory-MEDIANS_onlyWordForms_boundedVocab.pdf} & \includegraphics[width=0.25\textwidth]{../code/analyze_ngrams/visualize/figures/Naija-Adap-listener-surprisal-memory-MEDIANS_onlyWordForms_boundedVocab.pdf}
 \\ 
North Sami & Norwegian & Persian & Polish
 \\ 
\includegraphics[width=0.25\textwidth]{../code/analyze_ngrams/visualize/figures/North_Sami-listener-surprisal-memory-MEDIANS_onlyWordForms_boundedVocab.pdf} & \includegraphics[width=0.25\textwidth]{../code/analyze_ngrams/visualize/figures/Norwegian-listener-surprisal-memory-MEDIANS_onlyWordForms_boundedVocab.pdf} & \includegraphics[width=0.25\textwidth]{../code/analyze_ngrams/visualize/figures/Persian-listener-surprisal-memory-MEDIANS_onlyWordForms_boundedVocab.pdf} & \includegraphics[width=0.25\textwidth]{../code/analyze_ngrams/visualize/figures/Polish-listener-surprisal-memory-MEDIANS_onlyWordForms_boundedVocab.pdf}
 \\ 
Portuguese & Romanian & Russian & Serbian
 \\ 
\includegraphics[width=0.25\textwidth]{../code/analyze_ngrams/visualize/figures/Portuguese-listener-surprisal-memory-MEDIANS_onlyWordForms_boundedVocab.pdf} & \includegraphics[width=0.25\textwidth]{../code/analyze_ngrams/visualize/figures/Romanian-listener-surprisal-memory-MEDIANS_onlyWordForms_boundedVocab.pdf} & \includegraphics[width=0.25\textwidth]{../code/analyze_ngrams/visualize/figures/Russian-listener-surprisal-memory-MEDIANS_onlyWordForms_boundedVocab.pdf} & \includegraphics[width=0.25\textwidth]{../code/analyze_ngrams/visualize/figures/Serbian-listener-surprisal-memory-MEDIANS_onlyWordForms_boundedVocab.pdf}
 \\ 
Slovak & Slovenian & Spanish & Swedish
 \\ 
\includegraphics[width=0.25\textwidth]{../code/analyze_ngrams/visualize/figures/Slovak-listener-surprisal-memory-MEDIANS_onlyWordForms_boundedVocab.pdf} & \includegraphics[width=0.25\textwidth]{../code/analyze_ngrams/visualize/figures/Slovenian-listener-surprisal-memory-MEDIANS_onlyWordForms_boundedVocab.pdf} & \includegraphics[width=0.25\textwidth]{../code/analyze_ngrams/visualize/figures/Spanish-listener-surprisal-memory-MEDIANS_onlyWordForms_boundedVocab.pdf} & \includegraphics[width=0.25\textwidth]{../code/analyze_ngrams/visualize/figures/Swedish-listener-surprisal-memory-MEDIANS_onlyWordForms_boundedVocab.pdf}
 \\ 

\end{tabular}
	\caption{Medians (cont.)}
\end{table}

\begin{table}
\begin{tabular}{ccccccccccccccclll}
Thai & Turkish & Ukrainian & Urdu
 \\ 
\includegraphics[width=0.25\textwidth]{../code/analyze_ngrams/visualize/figures/Thai-Adap-listener-surprisal-memory-MEDIANS_onlyWordForms_boundedVocab.pdf} & \includegraphics[width=0.25\textwidth]{../code/analyze_ngrams/visualize/figures/Turkish-listener-surprisal-memory-MEDIANS_onlyWordForms_boundedVocab.pdf} & \includegraphics[width=0.25\textwidth]{../code/analyze_ngrams/visualize/figures/Ukrainian-listener-surprisal-memory-MEDIANS_onlyWordForms_boundedVocab.pdf} & \includegraphics[width=0.25\textwidth]{../code/analyze_ngrams/visualize/figures/Urdu-listener-surprisal-memory-MEDIANS_onlyWordForms_boundedVocab.pdf}
 \\ 
Uyghur & Vietnamese &  & 
 \\ 
\includegraphics[width=0.25\textwidth]{../code/analyze_ngrams/visualize/figures/Uyghur-Adap-listener-surprisal-memory-MEDIANS_onlyWordForms_boundedVocab.pdf} & \includegraphics[width=0.25\textwidth]{../code/analyze_ngrams/visualize/figures/Vietnamese-listener-surprisal-memory-MEDIANS_onlyWordForms_boundedVocab.pdf} &  & 
 \\ 

\end{tabular}
	\caption{Medians (cont.)}
\end{table}







%-- English, Korean, Russian
%-- UD$\_$Polish-LFG (released in 2.2, not included in original experiment) (13,744 sentences)
%-- character-level Russian
%\section{Character-Level Modeling}
%\section{Non-UD Dependency Treebanks}
%- other treebanks
%-- spoken Japanese (T{\"u}ba-J/S)
%-- another Vietnamese dependency treebank \citep{nguyen-bktreebank:-2017} (5,639 sentences)
%-- another Chinese dependency treebank LDC2012T05
%Due to the sizes of these treebanks, can also do experiment with full word forms.
%
%
%\section{Constituency Treebank}
%
%-- Penn treebank \citep{marcus-building-1993}
%
%-- spoken English (T{\"u}ba-E/S)
%
%-- spoken German (T{\"u}ba-D/S)
%
%-- Chinese treebank \citep{xue-chinese-2013}


\section{Formal Analysis and Proofs}
In this section, we prove the theorem described above.

We first make explicit how we formalize language processing for proving the theorem.

\paragraph{Language as a Stochastic Process}
We represent language as a stochastic process of words $\dots w_{-2} w_{-1} w_0 w_{1} w_{2} \dots$, extending indefinitely both into the past and into the future.
The symbols $w_i$ belong to a common set, representing the words of the language.\footnote{Could also be phonemes, sentences, ..., any other kind of unit.}

%We model the sequence as a probabilistic sequence; that is, given a context $w_{<t}$, the next word is distributed according to a distribution $p(w_t|w_{<t})$.

The assumption of infinite length is for mathematical convenience and does not affect the substance of our results:
As we restrict our attention to the processing of individual sentences, which have finite length, we will actually not make use of long-range and infinite contexts.

We make the assumption that this process is \emph{stationary}.
Formally, this means that the conditional distribution $P(w_t|w_{<t})$ does not depend on $t$, it only depends on the actual sequence $w_{<t}$.
Informally, this says that the process has no `internal clock', and that the statistical rules of the language do not change at the timescale we are interested in.
In reality, the statistical rules of language do change: They change as language changes over generations, and they also change between different situations -- e.g., depending on the interlocutor at a given point in time.
Given that we are interested in memory needs in the processing of \emph{individual sentences}, at a timescale of seconds or minutes, stationarity seems to be a reasonable assumption to make.





\begin{figure}
\includegraphics[width=0.45\textwidth]{figures/markov-condition.png}
	\caption{Illustration of (\ref{eq:listener-markov}). As the utterance unfolds, the listener maintains a memory state. After receiving word $w_t$, the listener computes their new memory state $m_t$ based on the previous memory state $m_{t-1}$ and the new word $w_t$.}\label{fig:listener-markov}
\end{figure}


\paragraph{Listener's Memory}
We now analyze memory from the perspective of the listener, who needs to maintain information about the past to predict the future.
As the speaker's utterance unfolds, the listener maintains a memory state $m_t$.

There are no assumptions about the memory architecture and the nature of its computations.
We only make a basic assumption about the flow of information (Figure~\ref{fig:listener-markov}):
At a given point in time, the listener's memory state $m_t$ is determined by the last word $w_t$, and the prior memory state $m_{t-1}$.
As a consequence, $m_t$ contains no information about the process beyond what is contained in the last word observed $w_{t-1}$ and in the memory state before that word was observed $m_{t-1}$.
This is formalized as a statement about conditional probabilities:
	\begin{equation}\label{eq:listener-markov}
		p(m_1| (w_{t})_{t \in \mathbb{Z}}, m_0)   = p(m_1 | m_0, w_1)
	\end{equation}
This says that $m_1$ contains no information about the utterances beyond what is contained in $m_0$ and $w_1$.	
As a consequence, the listener has no knowledge of the speaker's state beyond the information provided in their prior communication.
This is a simplification, as the listener could obtain information about the speaker from other sources, such as their common environment (weather, ...).
\mhahn{For the study of memory in sentence processing, this seems fair. Discuss this more.}

%
%First, we assume that the listener's internal state cannot depend on the future beyond its dependency on the past.
%Formally: 
%\begin{equation}\label{eq:listener-markov-1}
%m_t \bot w_{>t} | w_{\leq t}
%\end{equation}
%This means that the listener has no access to the speaker's state beyond what the speaker has already uttered.
%
%Second, we assume that $m_t$ contains no information about the past beyond what is contained in $w_{t-1}$ and $m_{t-1}$:
%\begin{equation}\label{eq:listener-markov-2}
%m_t \bot w_{<t} | w_{t-1}, m_{t-1}
%\end{equation}
%This means that any information about the past in $m_t$ has to be contained in $m_{t-1}$ -- formalizing the idea that a listener can only remember aspects of the past by keeping them in memory, and that memories of the past cannot `spontaneously' form later in the future.

%The listener can trade off memory and future surprisal:
%A listener who chooses to store less memory will exerience higher surprisal in the future.
%A listener can achieve minimal surprisal -- that is, the lowest average surprisal that any model could achieve by predicting the future from the past -- if and only if $m_t$ contains all predictive information about the future that is contained in the past.

%We now describe the memory-surprisal tradeoff. will describe this tradeoff, and show that listener memory is linked to locality in a way similar to speaker memory.
%Consider a listener who uses $J$ bits of memory on average.
%What can we say about the listener's surprisal?


\begin{thm}\label{prop:suboptimal}
	Let $T$ be any positive integer ($T \in \{1, 2, 3, ...\}$), and consider a listener using at most
	\begin{equation}\label{eq:memory}
		\sum_{t=1}^T t I_t
	\end{equation}
bits of memory on average.
Then this listener will incur surprisal at least
	$$H[w_t|w_{<t}] + \sum_{t > T} I_t$$
	on average.
\end{thm}
The proof is given in the appendix (REF).




\subsection{Proof of the Theorem}

We formalize a language as a stationary stochastic process $\dots w_{-2} w_{-1} w_0 w_{1} w_{2} \dots$, extending indefinitely both into the past and into the future.
The symbols $w_i$ belong to a common set, representing the words of the language.\footnote{Could also be phonemes, sentences, ..., any other kind of unit.}
We denote the listener's memory state at time $t$, after hearing $w_{<t} = ... w_{t-2} w_{t-1}$ by $m_t$.
As described above, we assume
\begin{equation}\label{eq:listener-markov}
	p(m_{t+1}| (w_{t'})_{t' \in \mathbb{Z}}, m_t)   = p(m_{t+1} | m_t, w_{t})
\end{equation}
that is, $m_{t+1}$ contains no information about the utterances beyond what is contained in $m_t$ and $w_{t}$.
As a consequence, the listener has no knowledge of the speaker's state beyond the information provided in their prior communication.


The average number of bits required to encode this state is $\operatorname{H}[m_t]$, which by assumption is at most $\sum_{t=1}^T t I_t$.
As the listener's predictions are made on the basis of her memory state, her average surprisal is at least $\operatorname{H}[w_t | m_t]$.
The difference between the listener's surprisal and optimal surprisal is thus at least $\operatorname{H}[w_t | m_t] - \operatorname{H}[w_t | w_{<t}]$.
By the assumption of stationarity, we can, for any positive integer $T$, rewrite this expression as
\begin{equation}\label{eq:byStation}
\operatorname{H}[w_t | m_t] - \operatorname{H}[w_t | w_{<t}] =  \frac{1}{T} \sum_{t'=1}^{T} \left(\operatorname{H}[w_{t'} | m_{t'}] - \operatorname{H}[w_{t'} | w_{<t'}]\right) 
\end{equation}
\begin{lemma}
For any positive integer $t$, the following inequality holds:
\begin{equation}
H[w_t | m_t] \geq H[w_t|w_{1 \dots t-1}, m_1]
\end{equation}
\end{lemma}
\begin{proof}[Proof of the Lemma]
	By Bayes' Theorem
\begin{align*}
	p(w_t|m_0, m_1, w_{0\dots t-1}) &= \frac{p(m_1|m_0, w_{0\dots t})}{p(m_1|m_0, w_{0\dots t-1})} \cdot p(w_t|m_0, w_{0\dots t-1})
\end{align*}
By Equation~\ref{eq:listener-markov}, the quotient on the RHS is equal to $1$, so
\begin{align*}
p(w_t|m_0, m_1, w_{0\dots t-1}) = p(w_t|m_0, w_{0\dots t-1})
\end{align*}
So we have a Markov chain
\begin{equation}
(w_t) \rightarrow (m_0, w_{0 \dots t-1})   \rightarrow   (m_1, w_{1 \dots t-1})
\end{equation}
Thus, by the Data Processing Inequality,
\begin{equation}
H[w_t| w_{1 \dots t-1}, m_{1}] \geq H[w_t|w_{0 \dots t-1}, m_0]
\end{equation}
Finally, iteratively applying this inequality, we get:
\begin{align*}
H[w_t | m_t] \geq H[w_t| w_{t-1}, m_{t-1}] \geq H[w_t| w_{t-2, t-1}, m_{t-2}] \geq ... \geq H[w_t|w_{1 \dots t-1}, m_1]
\end{align*}
\end{proof}
Plugging this inequality into Equation~\ref{eq:byStation} above:
\begin{align*}
\operatorname{H}[w_t | m_t] - \operatorname{H}[w_t | w_{<t}]& \geq \frac{1}{T} \sum_{t=1}^T ( \operatorname{H}[w_t|w_{1\dots t-1}, m_1] - \operatorname{H}[w_t | w_{1\dots t-1}, w_{\leq 0}]  )    \\
& = \frac{1}{T} \left(\operatorname{H}[w_{1\dots T} | m_1] - \operatorname{H}[w_{1\dots T} | w_{\leq 0}]\right)  \\
& = \frac{1}{T} \left(I[w_{1\dots T}|w_{\leq 0}] - I[w_{1\dots T}|m_1]\right) 
\end{align*}
The first term $I[w_{1\dots T}|w_{\leq 0}]$ can be rewritten in terms of $I_t$:
\begin{align*}
I[w_{1\dots T}|w_{\leq 0}] &= \sum_{i=1}^T \sum_{j=-1}^{-\infty} I[w_i, w_j | w_{j+1}...w_{i-1}] = \sum_{t=1}^T t I_t + T \sum_{t > T} I_t
\end{align*}
Therefore
\begin{align*}
\operatorname{H}[w_t | m_t] - \operatorname{H}[w_t | w_{<t}]& \geq \frac{1}{T} \left(\sum_{t=1}^T t I_t + T \sum_{t > T} I_t - I[w_{1\dots T}|m_1]\right) 
\end{align*}
$I[w_{1\dots T}|m_1]$ is at most $\operatorname{H}[m_1]$, which is at most $\sum_{t=1}^T t I_t$ by assumption. Thus, the expression above is bounded by
\begin{align*}
\operatorname{H}[w_t | m_t] - \operatorname{H}[w_t | w_{<t}]& \geq \frac{1}{T} \left(\sum_{t=1}^T t I_t + T \sum_{t > T} I_t - \sum_{t=1}^T t I_t\right) \\
&= \sum_{t > T} I_t
\end{align*}
Rearranging shows that the listener's surprisal is at least $\operatorname{H}[w_t|m_t] \geq \operatorname{H}[w_t | w_{<t}] + \sum_{t > T} I_t$, as claimed.
%\end{proof}


%Justify linear interpolation: The curve is convex, which is shown by `time-sharing': Use one code $\lambda$ fraction of times, and the other code $1-\lambda$ fraction of times.




\bibliographystyle{apalike}
\bibliography{literature}

\appendix




\end{document}






