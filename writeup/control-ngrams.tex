
%bounds.append(["alpha", float, 0.95, 1.0]) # + [x/20.0 for x in range(15, 21)])
%bounds.append(["gamma", int, 1, 2, 3, 4, 5, 8, 10, 15, 20, 25, 30]) # , 200, 300
%bounds.append(["delta", int, 0.1, 0.2, 0.5, 1.0 , 2.0, 3.0, 4.0, 5.0, 8.0, 10.0]) #, 1024]) #, 1024]) # 64, 128,
%bounds.append(["cutoff", int, 2,3,4,5,6,7,8,9,10]) #,7,8,9,10]) #, 1024]) #, 1024]) # 64, 128,
%

We use Interpolated Kneser-Ney.
Hyperparameters are tuned with the same strategy as for the neural network models.



\begin{table}
\begin{tabular}{ccccccccccccccclll}
Afrikaans & Amharic & Arabic & Armenian
 \\ 
\includegraphics[width=0.25\textwidth]{ngrams/figures/Afrikaans-listener-surprisal-memory-MEDIANS_onlyWordForms_boundedVocab.pdf} & \includegraphics[width=0.25\textwidth]{ngrams/figures/Amharic-Adap-listener-surprisal-memory-MEDIANS_onlyWordForms_boundedVocab.pdf} & \includegraphics[width=0.25\textwidth]{ngrams/figures/Arabic-listener-surprisal-memory-MEDIANS_onlyWordForms_boundedVocab.pdf} & \includegraphics[width=0.25\textwidth]{ngrams/figures/Armenian-Adap-listener-surprisal-memory-MEDIANS_onlyWordForms_boundedVocab.pdf}
 \\ 
Basque & Breton & Bulgarian & Buryat
 \\ 
\includegraphics[width=0.25\textwidth]{ngrams/figures/Basque-listener-surprisal-memory-MEDIANS_onlyWordForms_boundedVocab.pdf} & \includegraphics[width=0.25\textwidth]{ngrams/figures/Breton-Adap-listener-surprisal-memory-MEDIANS_onlyWordForms_boundedVocab.pdf} & \includegraphics[width=0.25\textwidth]{ngrams/figures/Bulgarian-listener-surprisal-memory-MEDIANS_onlyWordForms_boundedVocab.pdf} & \includegraphics[width=0.25\textwidth]{ngrams/figures/Buryat-Adap-listener-surprisal-memory-MEDIANS_onlyWordForms_boundedVocab.pdf}
 \\ 
Cantonese & Catalan & Chinese & Croatian
 \\ 
\includegraphics[width=0.25\textwidth]{ngrams/figures/Cantonese-Adap-listener-surprisal-memory-MEDIANS_onlyWordForms_boundedVocab.pdf} & \includegraphics[width=0.25\textwidth]{ngrams/figures/Catalan-listener-surprisal-memory-MEDIANS_onlyWordForms_boundedVocab.pdf} & \includegraphics[width=0.25\textwidth]{ngrams/figures/Chinese-listener-surprisal-memory-MEDIANS_onlyWordForms_boundedVocab.pdf} & \includegraphics[width=0.25\textwidth]{ngrams/figures/Croatian-listener-surprisal-memory-MEDIANS_onlyWordForms_boundedVocab.pdf}
 \\ 
Czech & Danish & Dutch & Estonian
 \\ 
\includegraphics[width=0.25\textwidth]{ngrams/figures/Czech-listener-surprisal-memory-MEDIANS_onlyWordForms_boundedVocab.pdf} & \includegraphics[width=0.25\textwidth]{ngrams/figures/Danish-listener-surprisal-memory-MEDIANS_onlyWordForms_boundedVocab.pdf} & \includegraphics[width=0.25\textwidth]{ngrams/figures/Dutch-listener-surprisal-memory-MEDIANS_onlyWordForms_boundedVocab.pdf} & \includegraphics[width=0.25\textwidth]{ngrams/figures/Estonian-listener-surprisal-memory-MEDIANS_onlyWordForms_boundedVocab.pdf}
 \\ 

\end{tabular}
	\caption{Medians (estimated using n-gram models): For each memory budget, we provide the median surprisal for real and random languages. Solid lines indicate sample medians for ngrams, dashed lines indicate 95 \% confidence intervals for the population median. Green: Random baselines; blue: real language; red: maximum-likelihood grammars fit to real orderings.}\label{tab:medians_ngrams}
\end{table}

\begin{table}
\begin{tabular}{ccccccccccccccclll}
English & Erzya & Estonian & Faroese
 \\ 
\includegraphics[width=0.25\textwidth]{../code/analyze_ngrams/visualize/figures/English-listener-surprisal-memory-MEDIANS_onlyWordForms_boundedVocab.pdf} & \includegraphics[width=0.25\textwidth]{../code/analyze_ngrams/visualize/figures/Erzya-Adap-listener-surprisal-memory-MEDIANS_onlyWordForms_boundedVocab.pdf} & \includegraphics[width=0.25\textwidth]{../code/analyze_ngrams/visualize/figures/Estonian-listener-surprisal-memory-MEDIANS_onlyWordForms_boundedVocab.pdf} & \includegraphics[width=0.25\textwidth]{../code/analyze_ngrams/visualize/figures/Faroese-Adap-listener-surprisal-memory-MEDIANS_onlyWordForms_boundedVocab.pdf}
 \\ 
Finnish & French & German & Greek
 \\ 
\includegraphics[width=0.25\textwidth]{../code/analyze_ngrams/visualize/figures/Finnish-listener-surprisal-memory-MEDIANS_onlyWordForms_boundedVocab.pdf} & \includegraphics[width=0.25\textwidth]{../code/analyze_ngrams/visualize/figures/French-listener-surprisal-memory-MEDIANS_onlyWordForms_boundedVocab.pdf} & \includegraphics[width=0.25\textwidth]{../code/analyze_ngrams/visualize/figures/German-listener-surprisal-memory-MEDIANS_onlyWordForms_boundedVocab.pdf} & \includegraphics[width=0.25\textwidth]{../code/analyze_ngrams/visualize/figures/Greek-listener-surprisal-memory-MEDIANS_onlyWordForms_boundedVocab.pdf}
 \\ 
Hebrew & Hindi & Hungarian & Indonesian
 \\ 
\includegraphics[width=0.25\textwidth]{../code/analyze_ngrams/visualize/figures/Hebrew-listener-surprisal-memory-MEDIANS_onlyWordForms_boundedVocab.pdf} & \includegraphics[width=0.25\textwidth]{../code/analyze_ngrams/visualize/figures/Hindi-listener-surprisal-memory-MEDIANS_onlyWordForms_boundedVocab.pdf} & \includegraphics[width=0.25\textwidth]{../code/analyze_ngrams/visualize/figures/Hungarian-listener-surprisal-memory-MEDIANS_onlyWordForms_boundedVocab.pdf} & \includegraphics[width=0.25\textwidth]{../code/analyze_ngrams/visualize/figures/Indonesian-listener-surprisal-memory-MEDIANS_onlyWordForms_boundedVocab.pdf}
 \\ 
Italian & Japanese & Kazakh & Korean
 \\ 
\includegraphics[width=0.25\textwidth]{../code/analyze_ngrams/visualize/figures/Italian-listener-surprisal-memory-MEDIANS_onlyWordForms_boundedVocab.pdf} & \includegraphics[width=0.25\textwidth]{../code/analyze_ngrams/visualize/figures/Japanese-listener-surprisal-memory-MEDIANS_onlyWordForms_boundedVocab.pdf} & \includegraphics[width=0.25\textwidth]{../code/analyze_ngrams/visualize/figures/Kazakh-Adap-listener-surprisal-memory-MEDIANS_onlyWordForms_boundedVocab.pdf} & \includegraphics[width=0.25\textwidth]{../code/analyze_ngrams/visualize/figures/Korean-listener-surprisal-memory-MEDIANS_onlyWordForms_boundedVocab.pdf}
 \\ 

\end{tabular}
	\caption{Medians (cont.)}
\end{table}

\begin{table}
\begin{tabular}{ccccccccccccccclll}
Kurmanji & Latvian & Maltese & Naija
 \\ 
\includegraphics[width=0.25\textwidth]{../code/analyze_ngrams/visualize/figures/Kurmanji-Adap-listener-surprisal-memory-MEDIANS_onlyWordForms_boundedVocab.pdf} & \includegraphics[width=0.25\textwidth]{../code/analyze_ngrams/visualize/figures/Latvian-listener-surprisal-memory-MEDIANS_onlyWordForms_boundedVocab.pdf} & \includegraphics[width=0.25\textwidth]{../code/analyze_ngrams/visualize/figures/Maltese-listener-surprisal-memory-MEDIANS_onlyWordForms_boundedVocab.pdf} & \includegraphics[width=0.25\textwidth]{../code/analyze_ngrams/visualize/figures/Naija-Adap-listener-surprisal-memory-MEDIANS_onlyWordForms_boundedVocab.pdf}
 \\ 
North Sami & Norwegian & Persian & Polish
 \\ 
\includegraphics[width=0.25\textwidth]{../code/analyze_ngrams/visualize/figures/North_Sami-listener-surprisal-memory-MEDIANS_onlyWordForms_boundedVocab.pdf} & \includegraphics[width=0.25\textwidth]{../code/analyze_ngrams/visualize/figures/Norwegian-listener-surprisal-memory-MEDIANS_onlyWordForms_boundedVocab.pdf} & \includegraphics[width=0.25\textwidth]{../code/analyze_ngrams/visualize/figures/Persian-listener-surprisal-memory-MEDIANS_onlyWordForms_boundedVocab.pdf} & \includegraphics[width=0.25\textwidth]{../code/analyze_ngrams/visualize/figures/Polish-listener-surprisal-memory-MEDIANS_onlyWordForms_boundedVocab.pdf}
 \\ 
Portuguese & Romanian & Russian & Serbian
 \\ 
\includegraphics[width=0.25\textwidth]{../code/analyze_ngrams/visualize/figures/Portuguese-listener-surprisal-memory-MEDIANS_onlyWordForms_boundedVocab.pdf} & \includegraphics[width=0.25\textwidth]{../code/analyze_ngrams/visualize/figures/Romanian-listener-surprisal-memory-MEDIANS_onlyWordForms_boundedVocab.pdf} & \includegraphics[width=0.25\textwidth]{../code/analyze_ngrams/visualize/figures/Russian-listener-surprisal-memory-MEDIANS_onlyWordForms_boundedVocab.pdf} & \includegraphics[width=0.25\textwidth]{../code/analyze_ngrams/visualize/figures/Serbian-listener-surprisal-memory-MEDIANS_onlyWordForms_boundedVocab.pdf}
 \\ 
Slovak & Slovenian & Spanish & Swedish
 \\ 
\includegraphics[width=0.25\textwidth]{../code/analyze_ngrams/visualize/figures/Slovak-listener-surprisal-memory-MEDIANS_onlyWordForms_boundedVocab.pdf} & \includegraphics[width=0.25\textwidth]{../code/analyze_ngrams/visualize/figures/Slovenian-listener-surprisal-memory-MEDIANS_onlyWordForms_boundedVocab.pdf} & \includegraphics[width=0.25\textwidth]{../code/analyze_ngrams/visualize/figures/Spanish-listener-surprisal-memory-MEDIANS_onlyWordForms_boundedVocab.pdf} & \includegraphics[width=0.25\textwidth]{../code/analyze_ngrams/visualize/figures/Swedish-listener-surprisal-memory-MEDIANS_onlyWordForms_boundedVocab.pdf}
 \\ 

\end{tabular}
	\caption{Medians (cont.)}
\end{table}

\begin{table}
\begin{tabular}{ccccccccccccccclll}
Thai & Turkish & Ukrainian & Urdu
 \\ 
\includegraphics[width=0.25\textwidth]{../code/analyze_ngrams/visualize/figures/Thai-Adap-listener-surprisal-memory-MEDIANS_onlyWordForms_boundedVocab.pdf} & \includegraphics[width=0.25\textwidth]{../code/analyze_ngrams/visualize/figures/Turkish-listener-surprisal-memory-MEDIANS_onlyWordForms_boundedVocab.pdf} & \includegraphics[width=0.25\textwidth]{../code/analyze_ngrams/visualize/figures/Ukrainian-listener-surprisal-memory-MEDIANS_onlyWordForms_boundedVocab.pdf} & \includegraphics[width=0.25\textwidth]{../code/analyze_ngrams/visualize/figures/Urdu-listener-surprisal-memory-MEDIANS_onlyWordForms_boundedVocab.pdf}
 \\ 
Uyghur & Vietnamese &  & 
 \\ 
\includegraphics[width=0.25\textwidth]{../code/analyze_ngrams/visualize/figures/Uyghur-Adap-listener-surprisal-memory-MEDIANS_onlyWordForms_boundedVocab.pdf} & \includegraphics[width=0.25\textwidth]{../code/analyze_ngrams/visualize/figures/Vietnamese-listener-surprisal-memory-MEDIANS_onlyWordForms_boundedVocab.pdf} &  & 
 \\ 

\end{tabular}
	\caption{Medians (cont.)}
\end{table}


